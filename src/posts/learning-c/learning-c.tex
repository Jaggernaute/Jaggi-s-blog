% Preamble
\documentclass[11pt]{article}

% Packages
\usepackage{array}
\usepackage[export]{adjustbox}
\usepackage{caption}
\usepackage[
  backend=biber,
  %style=reading
]{biblatex}
\usepackage[utf8]{inputenc}
\usepackage[T1]{fontenc}
\usepackage{hyperref}
\usepackage{graphicx}
\usepackage{geometry}
\usepackage{xcolor}
\usepackage{titlesec}
\usepackage{enumitem}
\usepackage{lmodern}
\usepackage{wasysym}
\usepackage{pgfplots}
\usepackage{svg}
\usepackage[off]{svg-extract}
\svgsetup{clean=true}
%\pdfsuppresswarningpagegroup=1
\usepackage{tikz}
\usepackage{tkz-euclide}

% Define a flag depending on compiler
\newif\ifhtml
\ifdefined\HCode
\htmltrue  % We're using make4ht/tex4ht
\else
\htmlfalse % We're using pdflatex or other
\fi

\addbibresource{electronics.bib}

\title{Learning C}
\author{Jaggi the nerdy fuzzball}
\date{\today}

% Document
\begin{document}
\maketitle
\ifhtml
\begin{center}
  \vspace{2em}
  \renewcommand{\arraystretch}{1.5}
  \begin{tabular}{
      >{\raggedright\arraybackslash}p{0.3\linewidth}
      >{\raggedright\arraybackslash}p{0.3\linewidth}
      >{\raggedright\arraybackslash}p{0.3\linewidth}
    }
    \href{../../index.html}{Blog Index} &
    \href{learning-c.pdf}{PDF
    Version}                            &
    \href{../../about.html}{About}              \\
    ~                                   & ~ & ~ \\
  \end{tabular}
\end{center}
\fi
\begin{flushright}
  \textit
  {Nevertheless, C retains the basic\\ philosophy that programmers know\\
    what they are doing; it only requires that\\ they state their
  intentions explicitly.}\par
  Brian W. Kernighan --- The C Programming Language --- 1978
\end{flushright}

\tableofcontents

\newpage
\begin{abstract}
  My notes about studying \textbf{C}, this will be full of \textbf{Not Safe
  For Memory} things (\textit{NSFM} stuff you don't wanna show your
  mom), fans of the almighty \textit{Krabby Patty} might want to stay
  away from this one.\\
  The whole purpose is to show what I learned, in which order, and
  where I have messed up. Maybe some low level guru (not in a
  pejorative way) might have some stuff to add on this, and
  \textbf{PLEASE} do so !
\end{abstract}

\begin{table}[h!]
  \centering
  \begin{tabular}{
      >{\centering\arraybackslash}p{0.35\textwidth}
      >{\raggedright\arraybackslash}m{0.65\textwidth}
    }
    \ifdefined\HCode
    \raisebox{-0.5\totalheight}{
      \includegraphics[height=15em,keepaspectratio]
      {src/assets/jaggi_teach.png}
    }
    \else
    \raisebox{-0.5\totalheight}{
      \includegraphics[height=9em,keepaspectratio]
      {src/assets/jaggi_teach_full.png}
    }
    \fi
    &
    \noindent
    If it hasn't segfaulted, yet you just need to run \textbf{ASan} a
    bit longer ...
  \end{tabular}
\end{table}

\section{Introduction} \label{sec:introduction}

\section{Setting up} \label{sec:setting}
First off, we need some packages, nothing colossal (we ain't doing no
Unreal \textbf{C++} stuff) and most of them might already be installed on your
distro --- Windows users might either want to struggle the \textbf{WSL}
way or sod off and go watch a \textbf{C\#} tutorial or something ---.\\
I'll use :
\begin{itemize}
  \item criterion
  \item gcc
  \item gcovr
  \item gnumake
  \item valgrind
\end{itemize}
This is a \textbf{Test}
\end{document}
